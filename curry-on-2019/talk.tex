\documentclass[aspectratio=169]{beamer}
\usepackage{minted}
\usemintedstyle{borland}
\usepackage{listings}
\usetheme{veit}
\title{Carp}
\subtitle{A Language for the 21st Century}
\date{\today}
\author{Veit Heller}
\institute{Port Zero}
\begin{document}
  \maketitle
  \section{Prologue}
  \begin{frame}{Obligatory Starter}
    \begin{itemize}
      \item Carp standard library maintainer
      \item CTO @ Port Zero
      \item Secretly a turtle
    \end{itemize}
  \end{frame}
  \begin{frame}{Questions \& Comments}
    This is a chess-timer talk. You’re welcome to ask questions at any time.
    \linebreak

    I can’t promise that I’ll be able to answer all of your questions.
    \linebreak

    I probably won’t get through my material.
  \end{frame}
  \section{I. Syntax \& other trivialities}
  \begin{frame}[fragile]
  \frametitle{Carp code}
    \begin{listing}[H]
      \caption{A silly Carp function}
      \begin{minted}{clojure}
; (type f)
; f : (Fn [(Ref (Array a)), Int, Int] a)
(defn f [x y z]
  @(Array.nth x (* y z)))
      \end{minted}
    \end{listing}
  \end{frame}
  \begin{frame}[fragile]
    \frametitle{Carp code}
    \begin{listing}[H]
      \caption{An associative array type, simplified.}
      \begin{minted}{clojure}
(deftype (AssocArray a b) [lst (Array (Pair a b))])
      \end{minted}
    \end{listing}
  \end{frame}
  \section{II. Semantics \& meaning}
  \begin{frame}{The goods}
    We have (non-hygienic) macros, type inference, and borrow checking.
    \linebreak

    The emphasis of the design is on simplicity while being pragmatic.
  \end{frame}
  \begin{frame}{What we’re still tuning}
    We’re working on good autogenerated documentation, dependency management,
    lifetimes, and, generally, tooling.
  \end{frame}
  \begin{frame}{Why?}
    There’s a vast amount of information flowing through the compiler, most
    of it implicit. \linebreak

    Reducing cognitive load is useful, but explorability is key.
  \end{frame}
  \section{III. Recap}
  \begin{frame}{Caveat}
    Carp is early stage software.
    \begin{itemize}
      \item[$\Rightarrow$] Small community, few packages
      \item[$\Rightarrow$] We’re less than a handful of maintainers
      \item[$\Rightarrow$] Insufficient documentation
      \item[$\Rightarrow$] May change under your feet
      \item[$\Rightarrow$] May blow up in your face!
    \end{itemize}
    Our first stable release—0.3.0—was released!
  \end{frame}
  \begin{frame}{References}
    \begin{itemize}
      \item Github: \texttt{https://github.com/carp-lang/carp}
      \item Erik: \texttt{https://github.com/eriksvedang}
      \item Chat: \texttt{https://gitter.im/carp-lang/carp}
      \item Docs: \texttt{http://carp-lang.github.io/Carp/core/core\_index.html}
      \item Blogs: \texttt{https://blog.veitheller.de} (sorry about that)
      \item Slides: \texttt{https://github.com/hellerve/carp\_talks}
      \item This talk, but different, shorter, and at clojuTRE: \texttt{https://www.youtube.com/watch?v=BQeG6fXMk28}
      \item Carp 0.3.0: \texttt{https://github.com/carp-lang/Carp/releases/tag/v0.3.0}
    \end{itemize}
  \end{frame}
  \begin{frame}{Epilogue}
    \Huge Thank you!
    \linebreak
    \linebreak
    \linebreak
    \small Questions?
    \linebreak
    \linebreak
    \tiny Slides at \texttt{https://github.com/hellerve/carp\_talks}
  \end{frame}
\end{document}
