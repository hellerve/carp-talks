\documentclass[aspectratio=169]{beamer}
\usepackage{minted}
\usepackage{listings}
\usetheme{metropolis}
\title{Carp}
\subtitle{A Language for the 21st Century}
\date{\today}
\author{Veit Heller}
\institute{Port Zero}
\begin{document}
  \maketitle
  \section{\texttt{man carp}}
  \begin{frame}{\texttt{man carp}}
    \begin{itemize}
      \item a Lisp-1
      \item type-inferred
      \item borrow-checked
      \item compiles to C
      \item for realtime applications
    \end{itemize}
  \end{frame}
  \begin{frame}{\texttt{man carp}}
    \begin{itemize}
      \item a Lisp-1
      \item type-inferred $\Rightarrow$ statically typed, at no extra charge
      \item borrow-checked $\Rightarrow$ no GC, at no extra charge
      \item compiles to C
      \item for realtime applications
    \end{itemize}
  \end{frame}
  \begin{frame}{\texttt{whence -v carp}}
    \begin{itemize}
      \item Haskell implements a Hindley-Milner type system and inference
      \begin{itemize}
        \item[$\Rightarrow$] You don’t have to spell types out!
      \end{itemize}
      \item Rust implements borrow checking
      \begin{itemize}
        \item[$\Rightarrow$] You don’t have to manually manage memory, even without a GC!
      \end{itemize}
    \end{itemize}
  \end{frame}
  \begin{frame}{\texttt{whence -v carp}}
    Let’s put those things together (after simplifying) and rejoice!
    \begin{itemize}
      \item[$\Rightarrow$] Also add some Lisp macro goodness and a near-seamless C FFI for good measure!
    \end{itemize}
  \end{frame}
  \section{\texttt{source carp}}
  \begin{frame}[fragile]
  \frametitle{\texttt{source carp}}
    \begin{listing}[H]
      \caption{A silly Carp function}
      \begin{minted}{clojure}
; (type f)
; f : (Fn [(Ref (Array a)), Int, Int] a)
(defn f [x y z]
  @(Array.nth x (* y z)))
      \end{minted}
    \end{listing}
  \end{frame}
  \begin{frame}[fragile]
    \frametitle{\texttt{source carp}}
    \begin{listing}[H]
      \caption{An associative array type, simplified.}
      \begin{minted}{clojure}
(deftype (AssocArray a b) [lst (Array (Pair a b))])
      \end{minted}
    \end{listing}
  \end{frame}
  \begin{frame}[fragile]
    \frametitle{\texttt{source carp}}
    \begin{listing}[H]
      \caption{A module for the associative array.}
      \begin{minted}{clojure}
(defmodule AssocArray
  (defn put [a k v]
    (let [list (lst a)
          pair (Pair k v)
          new-list (Array.push-back list pair)]
      (set-lst a new-list)))
)
      \end{minted}
    \end{listing}
  \end{frame}
  \section{open demo.live}
  \section{cmp clojure carp}
  \begin{frame}{\texttt{cmp clojure carp}}
    \begin{tabular}{ l | l }
      Carp & Clojure \\
      \hline
      \hline
      Statically typed & Dynamically typed \\ \hline
      Compiled & Kind of compiled? \\ \hline
      Works with C & Works with Java (and JS) \\ \hline
      \hline
      Modules & Namespaces \\ \hline
      Compile-time metaprogramming & full metaprogramming \\ \hline
      Interfaces & Protocols \\
    \end{tabular}
  \end{frame}
  \section{exit}
  \begin{frame}{\texttt{trap}}
    Carp is early stage software.
    \begin{itemize}
      \item[$\Rightarrow$] Small community, few packages
      \item[$\Rightarrow$] We’re less than a handful of maintainers
      \item[$\Rightarrow$] Insufficient documentation
      \item[$\Rightarrow$] May change under your feet
      \item[$\Rightarrow$] May blow up in your face!
    \end{itemize}
    We’re approaching the first stable release (0.3)
  \end{frame}
  \begin{frame}{\texttt{trap}}
    \begin{itemize}
      \item Github: \texttt{https://github.com/carp-lang/carp}
      \item Erik: \texttt{https://github.com/eriksvedang}
      \item Chat: \texttt{https://gitter.im/carp-lang/carp}
      \item Docs \& Blogs: \texttt{https://blog.veitheller.de} (sorry about that)
      \item Slides: \texttt{https://github.com/hellerve/carp\_talks}
      \item This talk, but different and at clojuTRE: \texttt{https://www.youtube.com/watch?v=BQeG6fXMk28}
    \end{itemize}
  \end{frame}
  \begin{frame}{\texttt{exit}}
    \Huge Thank you!
    \linebreak
    \linebreak
    \linebreak
    \small Questions?
    \linebreak
    \linebreak
    \tiny Slides at \texttt{https://github.com/hellerve/carp\_talks}
  \end{frame}
\end{document}
