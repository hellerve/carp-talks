\documentclass[aspectratio=169]{beamer}
\usepackage{minted}
\usepackage{listings}
\usepackage{tikz}
\usepackage{ulem}
\usetikzlibrary{arrows.meta, matrix}
\usetheme{metropolis}
\title{Carp}
\subtitle{A Language for the 21st Century}
\date{\today}
\author{Veit Heller}
\institute{Port Zero}
\begin{document}
  \maketitle
  \section{\texttt{whoami}}
  \begin{frame}{\texttt{whoami}}
    \begin{itemize}
      \item PL nerd
      \item CTO @ Port Zero
      \item Carp standard library maintainer
      \item Secretly a turtle
    \end{itemize}
  \end{frame}
  \section{\texttt{man carp}}
  \begin{frame}{\texttt{man carp}}
    \begin{itemize}
      \item a Lisp-1
      \item type-inferred
      \item borrow-checked
      \item compiles to C
      \item for realtime applications
    \end{itemize}
  \end{frame}
  \begin{frame}{\texttt{man carp}}
    \begin{itemize}
      \item a Lisp-1
      \item type-inferred $\Rightarrow$ statically typed, at no extra charge
      \item borrow-checked $\Rightarrow$ no GC, at no extra charge
      \item compiles to C
      \item for realtime applications
    \end{itemize}
  \end{frame}
  \begin{frame}{\texttt{whence -v carp}}
    \begin{itemize}
      \item Haskell implements a Hindley-Milner type system and inference
      \begin{itemize}
        \item[$\Rightarrow$] You don’t have to spell types out!
      \end{itemize}
      \item Rust implements borrow checking
      \begin{itemize}
        \item[$\Rightarrow$] You don’t have to manually manage memory, even without a GC!
      \end{itemize}
    \end{itemize}
  \end{frame}
  \begin{frame}{\texttt{whence -v carp}}
    Let’s put those things together (after simplifying) and rejoice!
    \begin{itemize}
      \item[$\Rightarrow$] Also add some Lisp macro goodness and a near-seamless C FFI for good measure!
    \end{itemize}
  \end{frame}
  \section{\texttt{source carp}}
  \begin{frame}[fragile]
  \frametitle{\texttt{source carp}}
    \begin{listing}[H]
      \caption{A silly Carp function}
      \begin{minted}{clojure}
; (type f)
; f : (Fn [(Ref (Array a)), Int, Int] a)
(defn f [x y z]
  @(Array.nth x (* y z)))
      \end{minted}
    \end{listing}
  \end{frame}
  \begin{frame}[fragile]
    \frametitle{\texttt{source carp}}
    \begin{listing}[H]
      \caption{An associative array type, simplified.}
      \begin{minted}{clojure}
(deftype (AssocArray a b) [lst (Array (Pair a b))])
      \end{minted}
    \end{listing}
  \end{frame}
%  \begin{frame}{\texttt{look hashmap}}
%      Carp has a typed (but generic) hashmap/dictionary type.
%  \end{frame}
%  \begin{frame}{\texttt{look hashmap}}
%      It is not a builtin type.
%  \end{frame}
%  \begin{frame}{\texttt{look hashmap}}
%      Let’s briefly look at a simple hashmap implementation
%      \begin{itemize}
%        \item A hash function determines the placement of an element in an array of arrays.
%        \item We append the element to the array inside the other to deal with hash collisions.
%        \item Lookup combines hashing and a linear search.
%      \end{itemize}
%  \end{frame}
%  \begin{frame}[fragile]
%  \frametitle{\texttt{look hashmap}}
%    \begin{figure}
%      \caption{A bucketed hashmap.}
%      \tikzset{
%            > = Stealth,
%Matrix/.style = {matrix of nodes,
%                 column 1/.style={nodes={draw, minimum size=10mm}},
%                 column 2/.style={nodes={draw, minimum size=10mm}},
%                 column 3/.style={nodes={draw, minimum size=10mm}},
%                 column 5/.style={nodes={draw, minimum size=10mm}},
%                 column 4/.style={font=\normalfont},
%                 column sep=-\pgflinewidth, row sep=0.7cm}
%      }
%      \begin{tikzpicture}
%        \matrix (m) [Matrix]
%        {
%        0 & 1 & 2 & ... & N \\
%        e0-0 & e1-0 & e2-0 & ... & eN-0 \\
%        e0-1 & e1-1 & & & \\
%        & e1-2 &  & & \\
%        };
%        \draw[<->] (m-1-1) edge (m-2-1)
%                   (m-1-2) edge (m-2-2)
%                   (m-1-3) edge (m-2-3)
%                   (m-1-5) edge (m-2-5)
%                   (m-2-1) edge (m-3-1)
%                   (m-2-2) edge (m-3-2)
%                   (m-3-2) edge (m-4-2);
%      \end{tikzpicture}
%    \end{figure}
%  \end{frame}
%  \begin{frame}[fragile]
%    \frametitle{\texttt{look hashmap}}
%    \begin{listing}[H]
%      \caption{The hashmap type, simplified.}
%      \begin{minted}{clojure}
%(deftype (Map a b) [buckets (Array (Array (Pair a b)))])
%      \end{minted}
%    \end{listing}
%  \end{frame}
%  \begin{frame}[fragile]
%    \frametitle{\texttt{look hashmap}}
%    \begin{listing}[H]
%      \caption{The hashmap module, with omissions.}
%      \begin{minted}{clojure}
%(defmodule Map
%   (def dflt-len 256)
%
%    (defn create []
%       (init (Array.repeat dflt-len Array.zero)))
%
%    (defn put [map key value]
%      ; ...
%    )
%)
%      \end{minted}
%    \end{listing}
%  \end{frame}
%  \begin{frame}[fragile]
%    \frametitle{\texttt{look hashmap}}
%    \begin{listing}[H]
%      \caption{Defining \texttt{put}.}
%      \begin{minted}{clojure}
%(defn put [map key value]
%   (let [buckets (buckets map)
%         len (Array.length buckets)
%         idx (Int.mod (hash key) len)
%         bucket @(Array.nth buckets idx)
%         pair (Pair.init @key @value)
%         new-bucket (Array.push-back bucket pair)
%         new-buckets (Array.aset @buckets
%                                 idx
%                                 new-bucket)
%        ]
%     (set-buckets @map new-buckets)))
%      \end{minted}
%    \end{listing}
%  \end{frame}
  \section{open demo.live}
  \section{cmp clojure carp}
  \begin{frame}{\texttt{cmp clojure carp}}
    \begin{tabular}{ l | l }
      Carp & Clojure \\
      \hline
      \hline
      Statically typed & Dynamically typed \\ \hline
      Compiled & Kind of compiled? \\ \hline
      Works with C & Works with Java (and JS) \\ \hline
      \hline
      Modules & Namespaces \\ \hline
      Compile-time metaprogramming & full metaprogramming \\ \hline
      Interfaces & Protocols \\
    \end{tabular}
  \end{frame}
  \section{exit}
  \begin{frame}{\texttt{trap}}
    Carp is early stage software.
    \begin{itemize}
      \item[$\Rightarrow$] Small community, few packages
      \item[$\Rightarrow$] We’re less than a handful of maintainers
      \item[$\Rightarrow$] Insufficient documentation
      \item[$\Rightarrow$] May change under your feet
      \item[$\Rightarrow$] May blow up in your face!
    \end{itemize}
    We’re approaching the first stable release (0.3)
  \end{frame}
%  \begin{frame}{\texttt{trap}}
%    Full disclosure: At runtime, there are no lists.
%  \end{frame}
%  \begin{frame}{\texttt{trap}}
%    Kiss \texttt{car}, \texttt{cdr}, \texttt{quote} and \texttt{eval} goodbye.
%  \end{frame}
%  \begin{frame}{\texttt{trap}}
%    At macro expansion, you have business as usual... at the expense of type safety.
%  \end{frame}
%  \begin{frame}[fragile]
%    \frametitle{\texttt{trap}}
%    \begin{listing}[H]
%      \caption{Conditionals as macros.}
%      \begin{minted}{clojure}
%(defmacro when [condition form]
%  (list 'if condition form (list)))
%
%(defmacro unless [condition form]
%  (list 'if condition (list) form))
%      \end{minted}
%    \end{listing}
%  \end{frame}
  \begin{frame}{\texttt{exit}}
    \Huge Thank you!
    \linebreak
    \linebreak
    \linebreak
    \small Questions?
    \linebreak
    \linebreak
    \tiny Slides at \texttt{https://github.com/hellerve/carp\_talks}
  \end{frame}
\end{document}
