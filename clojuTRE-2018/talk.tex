\documentclass{beamer}
\usepackage{enumitem}
\usepackage{minted}
\usepackage{listings}
\usepackage{ulem}
\usetheme{metropolis}           % Use metropolis theme
\newlist{arrowlist}{itemize}{1}
\setlist[arrowlist]{label=$\Rightarrow$}
\title{Carp}
\subtitle{A Language for the 21st Century}
\date{\today}
\author{Veit Heller}
\institute{Port Zero}
\begin{document}
  \maketitle
  \section{\texttt{whoami}}
  \begin{frame}{\texttt{whoami}}
    \begin{itemize}
      \item PL nerd
      \item CTO @ Port Zero
      \item Carp standard library maintainer
      \item Secretly a turtle
    \end{itemize}
  \end{frame}
  \section{\texttt{man carp}}
  \begin{frame}{\texttt{man carp}}
    \begin{itemize}
      \item a Lisp-1
      \item type-inferred
      \item borrow-checked
      \item compiles to C
      \item for realtime applications
    \end{itemize}
  \end{frame}
  \begin{frame}{\texttt{man carp}}
    \begin{itemize}
      \item a Lisp-1
      \item \sout{type-inferred} statically typed, at no extra charge
      \item \sout{borrow-checked} no GC, at not extra charge
      \item compiles to C
      \item for realtime applications
    \end{itemize}
  \end{frame}
  \begin{frame}{\texttt{whence -v carp}}
    \begin{itemize}
      \item Haskell implements a Hindley-Milner type system and inference
      \begin{arrowlist}
        \item You don’t have to spell types out anymore!
      \end{arrowlist}
      \item Rust implements borrow checking
      \begin{arrowlist}
        \item You don’t have to manually manage memory, even without a GC!
      \end{arrowlist}
    \end{itemize}
  \end{frame}
  \begin{frame}{\texttt{whence -v carp}}
    Let’s put those things together and rejoice!
    \begin{arrowlist}
      \item Also add some Lisp macro goodness and a near-seamless C FFI for good measure!
    \end{arrowlist}
  \end{frame}
  \section{\texttt{source carp}}
  \begin{frame}[fragile]
  \frametitle{\texttt{source carp}}
    \begin{listing}[H]
      \caption{A silly Carp function}
      \begin{minted}{clojure}
; (type f)
; f : (λ [(Ref (Array a)), Int, Int] a)
(defn f [x y z]
  @(Array.nth x (* y z)))
      \end{minted}
    \end{listing}
  \end{frame}
  \begin{frame}{\texttt{look hashmap}}
      Carp has a typed (but generic) hashmap/dictionary type.
  \end{frame}
  \begin{frame}{\texttt{look hashmap}}
      It is not a builtin type.
  \end{frame}
  \begin{frame}{\texttt{look hashmap}}
      Let’s briefly look at a simple hashmap implementation
      \begin{itemize}
        \item A hash function determines the placement of an element in an array of arrays.
        \item We append the element to the array inside the other to deal with hash collisions.
        \item Lookup combines hashing and a linear search.
      \end{itemize}
  \end{frame}
  \begin{frame}{\texttt{look hashmap}}
    Insert table here
  \end{frame}
  \begin{frame}[fragile]
    \frametitle{\texttt{look hashmap}}
    \begin{listing}[H]
      \caption{The hashmap type, simplified.}
      \begin{minted}{clojure}
(deftype (Map a b) [buckets (Array (Array (Pair a b)))])
      \end{minted}
    \end{listing}
  \end{frame}
  \begin{frame}[fragile]
    \frametitle{\texttt{look hashmap}}
    \begin{listing}[H]
      \caption{The hashmap module, with omissions.}
      \begin{minted}{clojure}
(defmodule Map
   (def dflt-len 256)

    (defn create []
       (init (Array.repeat dflt-len Array.zero)))

    (defn put [map key value]
      ; ...
    )
)
      \end{minted}
    \end{listing}
  \end{frame}
  \begin{frame}[fragile]
    \frametitle{\texttt{look hashmap}}
    \begin{listing}[H]
      \caption{Defining \texttt{put}.}
      \begin{minted}{clojure}
(defn put [map key value]
   (let [buckets (buckets map)
         len (Array.length buckets)
         idx (Int.mod (hash key) len)
         bucket @(Array.nth buckets idx)
         pair (Pair.init @key @value)
         new-bucket (Array.push-back bucket pair)
         new-buckets (Array.aset @b
                                 idx
                                 new-bucket)
        ]
     (set-buckets @map new-buckets)))
      \end{minted}
    \end{listing}
  \end{frame}
  \section{open demo.live}
  \section{exit}
  \begin{frame}{\texttt{trap}}
    Carp is early stage software.
    \begin{arrowlist}
      \item Small community, few packages
      \item We’re less than a handful of maintainers
      \item Insufficient documentation
      \item May change under your feet
      \item May blow up in your face!
    \end{arrowlist}
    We’re approaching the first stable release (0.2)
  \end{frame}
  \begin{frame}{\texttt{trap}}
    Full disclosure: At runtime, there are no lists.
  \end{frame}
  \begin{frame}{\texttt{trap}}
    Kiss \texttt{car}, \texttt{cdr}, \texttt{quote} and \texttt{eval} goodbye.
  \end{frame}
  \begin{frame}{\texttt{trap}}
    At macro expansion, you have business as usual... at the expense of type safety.
  \end{frame}
  \begin{frame}[fragile]
    \frametitle{\texttt{trap}}
    \begin{listing}[H]
      \caption{Conditionals as macros.}
      \begin{minted}{clojure}
(defmacro when [condition form]
  (list 'if condition form (list)))

(defmacro unless [condition form]
  (list 'if condition (list) form))
      \end{minted}
    \end{listing}
  \end{frame}
  \begin{frame}{\texttt{exit}}
    \Huge Thank you!
    \linebreak
    \linebreak
    \linebreak
    \small Questions?
  \end{frame}
\end{document}
